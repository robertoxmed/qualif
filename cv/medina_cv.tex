%%%%%%%%%%%%%%%%%%%%%%%%%%%%%%%%%%%%%%%%%
% University/School Laboratory Report
% LaTeX Template
% Version 3.1 (25/3/14)
%
% This template has been downloaded from:
% http://www.LaTeXTemplates.com
%
% Original author:
% Linux and Unix Users Group at Virginia Tech Wiki 
% (https://vtluug.org/wiki/Example_LaTeX_chem_lab_report)
%
% License:
% CC BY-NC-SA 3.0 (http://creativecommons.org/licenses/by-nc-sa/3.0/)
%
%%%%%%%%%%%%%%%%%%%%%%%%%%%%%%%%%%%%%%%%%

%----------------------------------------------------------------------------------------
%	PACKAGES AND DOCUMENT CONFIGURATIONS
%----------------------------------------------------------------------------------------

\documentclass{article}

\usepackage[version=3]{mhchem} % Package for chemical equation typesetting
\usepackage{siunitx} % Provides the \SI{}{} and \si{} command for typesetting SI units
\usepackage{graphicx} % Required for the inclusion of images
\usepackage{natbib} % Required to change bibliography style to APA
\usepackage{amsmath} % Required for some math elements 
\usepackage[utf8]{inputenc}
\usepackage[T1]{fontenc}
\usepackage{babel}
\usepackage{url}

\setlength\parindent{0pt} % Removes all indentation from paragraphs

\renewcommand{\labelenumi}{\alph{enumi}.} % Make numbering in the enumerate environment by letter rather than number (e.g. section 6)

%\usepackage{times} % Uncomment to use the Times New Roman font

%----------------------------------------------------------------------------------------
%	DOCUMENT INFORMATION
%----------------------------------------------------------------------------------------

\title{Dossier de qualification Maître de Conference \\ Section 27 : Informatique} % Title

\author{Roberto \textsc{Medina}} % Author name

\date{Session qualifications 2020} % Date for the report

\begin{document}

\maketitle % Insert the title, author and date

%----------------------------------------------------------------------------------------
%	SECTION 1
%----------------------------------------------------------------------------------------

\section{Identité}

Date de naissance : 2 Avril 1992

Nationalité : franco-équatorienne


Addresse : 314 rue de Charenton, 75012 Paris

Email : roberto.medina-bonilla@inria.fr

Site web : \url{https://roberto-medina.eu/}

\subsection{Formations}

\begin{center}
\fbox{2019 $\circ$ \textbf{Doctorat en Informatique} $\circ$ TELECOM ParisTech}
\end{center}

\textit{Titre} :\textbf{ Déploiement de systèmes à flots de données en criticité mixte pour architectures 
multi-coeurs}\footnote{\url{http://www.theses.fr/2019SACLT004}}

\textit{Directeur} : Laurent \textsc{Pautet}, Professeur, TELECOM ParisTech.

\textit{Encadrant} : Etienne \textsc{Borde}, Maître de conférences, TELECOM ParisTech.

\textit{Période} : Octobre 2015 - Janvier 2019

\textit{Composition du jury} :
\begin{itemize}
	\item Alix \textsc{Munier-Kordon}, Professeure, Sorbonne Université (LIP6), Présidente
	\item Liliana \textsc{Cucu-Grosjean}, Chargée de recherche, INRIA de Paris (Kopernic), Rapporteuse
	\item Laurent \textsc{George}, Professeur, ESIEE (LIGM), Rapporteur
	\item Arvind \textsc{Easwaran}, Maître de conférences, Nanyang Technological University, Examinateur
	\item Eric \textsc{Goubault}, Professeur, Ecole Polytechnique (LIX), Examinateur
	\item Emmanuel \textsc{Ledinot}, Responsable de recherche et technologie, Dassault Aviation, 
	Examinateur
	\item Isabelle \textsc{Puaut}, Professeure, Université Rennes 1 (IRISA), Examinatrice
\end{itemize}

\textit{Laboratoire} : Equipe ACES du LTCI, TELECOM ParisTech

\textit{Financement} : Chaire Ingénierie des Systèmes 
Complexes\footnote{\url{https://www.telecom-paris.fr/fr/recherche/recherche-partenariale/les-chaires-de-recherche/ingenierie-des-systemes-complexes}}

\begin{center}
	\fbox{2015 $\circ$ \textbf{Master Informatique} $\circ$  Sobronne Universités (Paris 6)}
\end{center}

\textit{Specialité} : Systèmes et Applications Répartis (SAR)

\textit{Mémoire} : \textbf{Conception et Développement de Systèmes Critiques sur Architectures 
	Multi-C\oe{}urs}

\textit{Encadrants} : Thomas \textsc{Robert} \& Julien \textsc{Sopena} 

\textit{Mention} : Bien

\textit{Période} : 	Septembre 2013 - Septembre 2015



\begin{center}
	\fbox{2013 $\circ$ \textbf{Licence Math.-Info.} $\circ$ Université Claude Bernard (Lyon 1)}
\end{center}

\textit{Mention} : Assez Bien

\textit{Période} : 	Septembre 2010 - Juillet 2013

\subsection{Fonctions Occupées}


\begin{table}[h]
	\begin{tabular}{|c|c|c|c|}
		\hline
		Institution       & Poste          & Début      & Fin        \\ \hline
		INRIA Paris       & Post-doctorant & 01/02/2019 &            \\ \hline
		TELECOM ParisTech & Doctorant      & 01/10/2015 & 31/01/2019 \\ \hline
		TELECOM ParisTech & Stagiaire      & 01/05/2015 & 30/09/2015 \\ \hline
	\end{tabular}
\end{table}

\textit{Situation actuelle} : Post-doctorant à l'INRIA de Paris. Équipe Kopernic.

%----------------------------------------------------------------------------------------
%	SECTION 2
%----------------------------------------------------------------------------------------

\section{Publications}

\subsection{Articles}

\begin{itemize}
	\item \textbf{Conférences internationales avec comité de lecture: articles longs}
	
	[1] Medina R., Borde E. et Pautet L.,  \textit{"Scheduling Multi-Periodic Mixed-Criticality DAGs on 
	Multi-Core Architectures"}, 39th IEEE Real-Time Systems Symposium (RTSS), Nashville, États-Unis, 2018

	[2] Medina R., Borde E. et Pautet L.,  \textit{"Availability enhancement and analysis for mixed-criticality 
	systems on multi-core"}, Design, Automation \& Test in Europe Conference \& Exhibition (DATE), Dresde, 
	Allemagne, 2018
	
	[3] Medina R., Borde E. et Pautet L.,  \textit{"Directed acyclic graph scheduling for mixed-criticality 
	systems"}, International Conference on Reliable Software Technologies (Ada-Europe), Vienne, Autriche, 
	2017
	
	\item \textbf{Conférences internationales avec comité de lecture: articles courts}
	
	[4] Medina R., Borde E. et Pautet L.,  \textit{"Availability analysis for synchronous data-flow graphs in 
	mixed-criticality systems"}, 11th IEEE Symposium on Industrial Embedded Systems (SIES), Cracovie, 
	Pollogne, 	2016
	
	[5] Medina R., Cucu L.,  \textit{"Work-in-Progress: System-wide DVFS for real-time
systems with 
	probabilistic parameters"}, 40th IEEE Real-Time Systems Symposium (RTSS), 2019
	
\end{itemize}

\rule{\textwidth}{0.4pt}

\begin{itemize}
	\item \textbf{Revues audience internationale}
	
	[6] Medina R., Borde E. et Pautet L.,  \textit{"Generalized Mixed-Criticality Scheduling for Periodic 
	Directed Acyclic Graphs"}, IEEE Transations on Computers, \textbf{\textit{Sous revue}}
	
\end{itemize}

\subsection{Communications effectuées}

\begin{itemize}
	\item \textbf{Manifestations d'audience internationale}
	
	$\circ$ "System-wide power management for real-time systems", 10th Real-Time Scheduling Open 
	Problems Seminar (RTSOPS), INRIA, Paris, Juillet 2019
	
	\item \textbf{Séminaires internes}
	
	$\circ$ "Mixed-Criticality scheduling for
	Directed Acyclic Graphs", Junior Seminar, INRIA, Paris, Mars 
	2019
	
	$\circ$ "Scheduling Mixed-Criticality DAGs on multi-core
architectures", Journées du LTCI, TELECOM 
	ParisTech, Paris, Octobre 2018
	
	$\circ$ "Scheduling of multi-periodic DAGs on multi-core for mixed-criticality", Séminaire Worsted, CEA 
	List, Palaisseau, Novembre 2018
\end{itemize}

%----------------------------------------------------------------------------------------
%	SECTION 3
%----------------------------------------------------------------------------------------

\section{Activités}


%----------------------------------------------------------------------------------------
\subsection{Enseignement}
%----------------------------------------------------------------------------------------

Pendant mon doctorat à TELECOM ParisTech j'ai assuré différents Travaux Pratiques (TP) pour une durée de 
32 heures/an pendant trois ans. Encadrer des TPs n'était pas obligatoire pendant mes années de doctorat 
mais donnait lieu à un avenant sur le contrat doctoral. Comme mon objectif est aussi d'obtenir la qualification 
après mon doctorat, donner des TPs me permettait d'avoir une première expérience dans l'enseignement 
supérieur.
\vspace{.5cm}

L'encadrement des TPs pour l'UE \guillemotleft\ 
Système d'exploitation et langage C 
\guillemotright\footnote{\url{https://inf104.wp.imt.fr/}} se déroulait au sein de TELECOM ParisTech pour 
des 
élèves de première année de cycle d'ingénieur (BAC+3) de l'école. Ces TPs se déroulaient pendant le 
premier semestre de l'année universitaire pour des groupes d'environ 15-20 personnes et prenaient environ 
26 heures en total. Les TPs de cette UE s'organisaient dans deux parties principales. La première partie 
abordait des notions de programmation en C, comme la manipulation de chaînes de caractères, 
boucles, écriture sur des fichiers, allocation mémoire dynamique et quelques structures de données. La 
deuxième partie abordait des notions de programmation système, comme la création/synchronisation de 
processus lourds, l'utilisation de sémaphores et de signaux.
\vspace{.5cm}

Les 32 heures d'enseignement étaient complétées par 6 heures d'encadrement des TPs pour l'UE 
\guillemotleft\ Systémes Temps Réel Embarqués Critiques 
\guillemotright\footnote{\url{https://strec.wp.imt.fr/}}. Cette UE se déroulait à TELECOM ParisTech mais 
était destinée à des élèves de Master 2 d'autres universités comme Paris 6 (Master SAR) et Paris 11 (Master 
SETI/COMASIC). Ces TPs se déroulaient pendant le premier semestre de l'année universitaire pour des 
groupes d'entre 10 et 20 personnes. Ces TPs abordaient des notions de systèmes temps-réel avec de la 
programmation RT-POSIX, ainsi que la manipulation d'outils de modélisation et génération de code.

%----------------------------------------------------------------------------------------
\subsection{Recherche}
%----------------------------------------------------------------------------------------

\textbf{Optimisation de la consommation énergétique pour des systèmes embarqués temps-réel}

Dans le cadre de mon post-doctorat à l'INRIA de Paris au sein de l'équipe Kopernic, je me suis intéressé à 
des problèmes de consommation énergétique pour des systèmes temps-réel. Ce problème est 
particulièrement intéressant puisque la plupart des systèmes temps-réels sont utilisés dans des 
architectures embarquées alimentées par des batteries. Limiter la consommation énergétique tout en 
respectant les contraintes temporelles de ses systèmes représente une vrai nécessité dans le monde 
académique et industriel.

Les méthodes actuelles pour limiter la consommation énergétique sur les systèmes temps-réels sont 
souvent basées sur le changement de fréquence du processeur. Les processeurs modernes offrent la 
possibilité de réduire la tension fournie aux c\oe{}urs, ce qui réduit leur fréquence d'exécution et diminue la 
consommation énergétique. Cependant, avec une fréquence moins élevée, le temps d'exécution  des 
programmes augmente. La problématique est donc de trouver des fréquences qui permettent de réduire la 
consommation en énergie tout en respectant les contraintes temporelles des programmes.


Le changement de fréquence du processur est possible dû au fait que les systèmes temps-réels sont 
dimensionnés en respectant le pire cas d'exécution pour les programmes. Néanmoins, le pire cas arrive 
dans des situation extrêmement rares et sans mécanisme de réduction énergétique, ces ressources sont 
gaspillées.
\vspace{.5cm}


(Résultats obtenus)

Pendant mon post-doctorat 


(Diffusion résultats)


(Encadrement)

\textbf{Déploiement de systèmes à flots de données pour architectures multi-c\oe{}urs en criticité mixte}

I spent three years working at Télécom ParisTech under the supervision of Laurent Pautet 0 and Etienne 
Borde 1 . The
research activity was focused on mixed-criticality (MC) systems that are data-driven (i.e. the system is 
constrained to
respect data dependencies) and executed on multi-core architectures. The MC execution model is a 
real-time model were
tasks with different criticalities share a common execution platform. These tasks have stringent time 
constraints and are
modeled as periodic events that need to be completed before a certain deadline. While there have been 
contributions
proposing scheduling policies for MC tasks on multi-core architectures, few works have considered data 
dependencies
and tend to over-dimension systems. Two main objectives were tackled during my thesis: (i) the definition of 
new efficient
scheduling methods for MC and data-driven systems; and (ii) the analysis of availability for the non-critical 
components.
The scheduling problem of MC tasks with data dependencies is very difficult (NP-hard). The difficulty comes 
from the
fact that MC systems need to be scheduled under different execution modes, i.e. deadlines, periods, data 
dependencies
need to be respected for all these execution modes. At the same time, these constraints need to be respected 
when the
system changes its execution mode. The modeling of our research context led to the definition of a task 
model called
Mixed-Criticality Directed Acyclic Graph (MC-DAG). Scheduling approaches capable of allocating MC-DAGs 
into multi-core
architectures have been developed in the literature but show relevant short-comes in terms of acceptance 
rate (i.e. number
of schedulable tasks sets) and resource usage. In order to lift the limitations we have characterized a 
sufficient property to
have safe mode transitions on MC systems. When this condition is respected, the system always respects 
deadlines, even
when a mode switch occurs. Building upon this property we defined a meta-heuristic to schedule MC-DAGs 
on multi-core
architectures. A first improvement considered, was the relaxation on the execution of the most critical tasks 
allowing us to
reduce the response time of the less-critical tasks which improved schedulability. Experimental results 
presenting our gains
in terms of acceptance rate were first presented in [RM3]. The generalization to support multiple 
multi-periodic MC-DAGs
on a shared multi-core architecture was presented in [RM1]. To support this multi-periodic case, existing 
works follow a
federated approach: each MC-DAG will be executed in an exclusive cluster of cores. Our solution follows a 
global approach:
vertices of different MC-DAGs can be scheduled on all cores. Again experimental results presented in [RM1] 
demonstrated
that our improvements were effective. The final extension to our works is capable of computing schedulers 
for MC systems
with an arbitrary number of criticality levels. This generalization is presented in my thesis manuscript [RM5] 
and will be
submitted including additional results to a journal in the upcoming weeks [RM6].
The second topic we were interested in, was the availability of non-critical tasks executing in a MC system. 
Even if non-
criticality tasks have low or no impact on the system’s safety, their execution is important for the system’s 
usability. The
industry has criticized the MC model because non-critical tasks are often dropped to ensure the 
schedulability of critical
tasks in all operational modes. Recent works on MC have proposed different approaches to overcome this 
limitation by
extending the most common task model of the literature. In our case, we wanted to perform availability 
evaluations and
incorporate availability enhancements that are compatible with the scheduling methods we defined. To 
perform availability
evaluations we need to provide sufficient information about the system behavior which led us to the inclusion 
of a fault
model and a recovery mechanism for the non-critical tasks. When the task model has this information 
analytical analysis
can be performed. As expected, the most common degradation model of MC scheduling delivers a low 
availability rate for
non-critical tasks. For this reason, had planned to incorporate availability enhancements. A first 
enhancements was the
definition of a more precise fault propagation model, where only data-dependent tasks are interrupted and 
a mode transition
to a higher criticality mode occurs when a critical task needs more processing resources. Our evaluations 
performed in
[RM2] showed that our fault propagation model was effective. Nevertheless, a common practice in the 
design of safety-
critical systems consists in incorporating “robust” tasks or follow certain “safety” design patterns. In [RM2] 
we demonstrated
that weakly-hard real-time tasks can be used as a mean to improve availability. In order to obtain availability 
rate when this
type of tasks are incorporated in the system, we need to perform system simulations. I defined translations 
rules to obtain
probabilistic automata. We first presented these transformation rules for MC systems executing in 
mono-core processors
on [RM4], new rules were presented in [RM2] and the generalization of the rules for multiple criticality 
levels are detailed in
[RM5].
All the abovementioned contributions were integrated into an open source framework [RM7]. This 
framework also includes
an unbiased generator of MC-DAGs to test our scheduling algorithms and compare them to existing works. 
The proposed
contributions are being integrated in the RAMSES tool, a widely used tool for research projects at Télécom 
ParisTech.
Extensions about my works related to the scheduling of MC-DAGs are now being implemented by a 
post-doctoral researcher
at Télécom ParisTech.



\end{document}