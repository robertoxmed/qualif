%%%%%%%%%%%%%%%%%%%%%%%%%%%%%%%%%%%%%%%%%
% University/School Laboratory Report
% LaTeX Template
% Version 3.1 (25/3/14)
%
% This template has been downloaded from:
% http://www.LaTeXTemplates.com
%
% Original author:
% Linux and Unix Users Group at Virginia Tech Wiki 
% (https://vtluug.org/wiki/Example_LaTeX_chem_lab_report)
%
% License:
% CC BY-NC-SA 3.0 (http://creativecommons.org/licenses/by-nc-sa/3.0/)
%
%%%%%%%%%%%%%%%%%%%%%%%%%%%%%%%%%%%%%%%%%

%----------------------------------------------------------------------------------------
%	PACKAGES AND DOCUMENT CONFIGURATIONS
%----------------------------------------------------------------------------------------

\documentclass{article}

\usepackage[version=3]{mhchem} % Package for chemical equation typesetting
\usepackage{siunitx} % Provides the \SI{}{} and \si{} command for typesetting SI units
\usepackage{graphicx} % Required for the inclusion of images
\usepackage{natbib} % Required to change bibliography style to APA
\usepackage{amsmath} % Required for some math elements 
\usepackage[utf8]{inputenc}
\usepackage[T1]{fontenc}
\usepackage{babel}
\usepackage{url}

\setlength\parindent{0pt} % Removes all indentation from paragraphs

\renewcommand{\labelenumi}{\alph{enumi}.} % Make numbering in the enumerate environment by letter rather than number (e.g. section 6)

%\usepackage{times} % Uncomment to use the Times New Roman font

%----------------------------------------------------------------------------------------
%	DOCUMENT INFORMATION
%----------------------------------------------------------------------------------------

\title{Dossier de qualification Maître de Conference \\ Section 27 : Informatique} % Title

\author{Roberto \textsc{Medina}} % Author name

\date{Session qualifications 2020} % Date for the report

\begin{document}

\maketitle % Insert the title, author and date

%----------------------------------------------------------------------------------------
%	SECTION 1
%----------------------------------------------------------------------------------------

\section{Identité}

Date de naissance : 2 Avril 1992

Nationalité : franco-équatorienne


Addresse : 314 rue de Charenton, 75012 Paris

Email : roberto.medina-bonilla@inria.fr

Site web : \url{https://roberto-medina.eu/}

\subsection{Formations}

\begin{center}
\fbox{2019 $\circ$ \textbf{Doctorat en Informatique} $\circ$ TELECOM ParisTech}
\end{center}

\textit{Titre} :\textbf{ Déploiement de systèmes à flots de données en criticité mixte pour architectures 
multi-coeurs}\footnote{\url{http://www.theses.fr/2019SACLT004}}

\textit{Directeur} : Laurent \textsc{Pautet}, Professeur, TELECOM ParisTech.

\textit{Encadrant} : Etienne \textsc{Borde}, Maître de conférences, TELECOM ParisTech.

\textit{Période} : Octobre 2015 - Janvier 2019

\textit{Composition du jury} :
\begin{itemize}
	\item Alix \textsc{Munier-Kordon}, Professeure, Sorbonne Université (LIP6), Présidente
	\item Liliana \textsc{Cucu-Grosjean}, Chargée de recherche, INRIA de Paris (Kopernic), Rapporteuse
	\item Laurent \textsc{George}, Professeur, ESIEE (LIGM), Rapporteur
	\item Arvind \textsc{Easwaran}, Maître de conférences, Nanyang Technological University, Examinateur
	\item Eric \textsc{Goubault}, Professeur, Ecole Polytechnique (LIX), Examinateur
	\item Emmanuel \textsc{Ledinot}, Responsable de recherche et technologie, Dassault Aviation, 
	Examinateur
	\item Isabelle \textsc{Puaut}, Professeure, Université Rennes 1 (IRISA), Examinatrice
\end{itemize}

\textit{Laboratoire} : Equipe ACES du LTCI, TELECOM ParisTech

\textit{Financement} : Chaire Ingénierie des Systèmes 
Complexes\footnote{\url{https://www.telecom-paris.fr/fr/recherche/recherche-partenariale/les-chaires-de-recherche/ingenierie-des-systemes-complexes}}

\begin{center}
	\fbox{2015 $\circ$ \textbf{Master Informatique} $\circ$  Sobronne Universités (Paris 6)}
\end{center}

\textit{Specialité} : Systèmes et Applications Répartis (SAR)

\textit{Mémoire} : \textbf{Conception et Développement de Systèmes Critiques sur Architectures 
	Multi-C\oe{}urs}

\textit{Encadrants} : Thomas \textsc{Robert} \& Julien \textsc{Sopena} 

\textit{Mention} : Bien

\textit{Période} : 	Septembre 2013 - Septembre 2015



\begin{center}
	\fbox{2013 $\circ$ \textbf{Licence Math.-Info.} $\circ$ Université Claude Bernard (Lyon 1)}
\end{center}

\textit{Mention} : Assez Bien

\textit{Période} : 	Septembre 2010 - Juillet 2013

\subsection{Fonctions Occupées}


\begin{table}[h]
	\begin{tabular}{|c|c|c|c|}
		\hline
		Institution       & Poste          & Début      & Fin        \\ \hline
		INRIA Paris       & Post-doctorant & 01/02/2019 &            \\ \hline
		TELECOM ParisTech & Doctorant      & 01/10/2015 & 31/01/2019 \\ \hline
		TELECOM ParisTech & Stagiaire      & 01/05/2015 & 30/09/2015 \\ \hline
	\end{tabular}
\end{table}

\textit{Situation actuelle} : Post-doctorant à l'INRIA de Paris. Équipe Kopernic.

%----------------------------------------------------------------------------------------
%	SECTION 2
%----------------------------------------------------------------------------------------

\section{Publications}

\subsection{Articles}

\begin{itemize}
	\item \textbf{Conférences internationales avec comité de lecture: articles longs}
	
	[1] Medina R., Borde E. et Pautet L.,  \textit{"Scheduling Multi-Periodic Mixed-Criticality DAGs on 
	Multi-Core Architectures"}, 39th IEEE Real-Time Systems Symposium (RTSS), Nashville, États-Unis, 2018

	[2] Medina R., Borde E. et Pautet L.,  \textit{"Availability enhancement and analysis for mixed-criticality 
	systems on multi-core"}, Design, Automation \& Test in Europe Conference \& Exhibition (DATE), Dresde, 
	Allemagne, 2018
	
	[3] Medina R., Borde E. et Pautet L.,  \textit{"Directed acyclic graph scheduling for mixed-criticality 
	systems"}, International Conference on Reliable Software Technologies (Ada-Europe), Vienne, Autriche, 
	2017
	
	\item \textbf{Conférences internationales avec comité de lecture: articles courts}
	
	[4] Medina R., Borde E. et Pautet L.,  \textit{"Availability analysis for synchronous data-flow graphs in 
	mixed-criticality systems"}, 11th IEEE Symposium on Industrial Embedded Systems (SIES), Cracovie, 
	Pollogne, 	2016
	
	[5] Medina R., Cucu L.,  \textit{"Work-in-Progress: System-wide DVFS for real-time
systems with 
	probabilistic parameters"}, 40th IEEE Real-Time Systems Symposium (RTSS), 2019
	
\end{itemize}

\rule{\textwidth}{0.4pt}

\begin{itemize}
	\item \textbf{Revues audience internationale}
	
	[6] Medina R., Borde E. et Pautet L.,  \textit{"Generalized Mixed-Criticality Scheduling for Periodic 
	Directed Acyclic Graphs"}, IEEE Transations on Computers, \textbf{\textit{Sous revue}}
	
\end{itemize}

\subsection{Communications effectuées}

\begin{itemize}
	\item \textbf{Manifestations d'audience internationale}
	
	$\circ$ "System-wide power management for real-time systems", 10th Real-Time Scheduling Open 
	Problems Seminar (RTSOPS), INRIA, Paris, Juillet 2019
	
	\item \textbf{Séminaires internes}
	
	$\circ$ "Mixed-Criticality scheduling for
	Directed Acyclic Graphs", Junior Seminar, INRIA, Paris, Mars 
	2019
	
	$\circ$ "Scheduling Mixed-Criticality DAGs on multi-core
architectures", Journées du LTCI, TELECOM 
	ParisTech, Paris, Octobre 2018
	
	$\circ$ "Scheduling of multi-periodic DAGs on multi-core for mixed-criticality", Séminaire Worsted, CEA 
	List, Palaisseau, Novembre 2018
\end{itemize}

%----------------------------------------------------------------------------------------
%	SECTION 3
%----------------------------------------------------------------------------------------

\section{Activités}


%----------------------------------------------------------------------------------------
\subsection{Enseignement}
%----------------------------------------------------------------------------------------

Pendant mon doctorat à TELECOM ParisTech j'ai assuré différents Travaux Pratiques (TP) pour une durée de 
32 heures/an pendant trois ans. Encadrer des TPs n'était pas obligatoire pendant mes années de doctorat 
mais donnait lieu à un avenant sur le contrat doctoral. Comme mon objectif est aussi d'obtenir la qualification 
après mon doctorat, donner des TPs me permettait d'avoir une première expérience dans l'enseignement 
supérieur.
\vspace{.5cm}

L'encadrement des TPs pour l'UE \guillemotleft\ 
Système d'exploitation et langage C 
\guillemotright\footnote{\url{https://inf104.wp.imt.fr/}} se déroulait au sein de TELECOM ParisTech pour 
des 
élèves de première année de cycle d'ingénieur (BAC+3) de l'école. Ces TPs se déroulaient pendant le 
premier semestre de l'année universitaire pour des groupes d'environ 15-20 personnes et prenaient environ 
26 heures en total. Les TPs de cette UE s'organisaient dans deux parties principales. La première partie 
abordait des notions de programmation en C, comme la manipulation de chaînes de caractères, 
boucles, écriture sur des fichiers, allocation mémoire dynamique et quelques structures de données. La 
deuxième partie abordait des notions de programmation système, comme la création/synchronisation de 
processus lourds, l'utilisation de sémaphores et de signaux.
\vspace{.5cm}

Les 32 heures d'enseignement étaient complétées par 6 heures d'encadrement des TPs pour l'UE 
\guillemotleft\ Systémes Temps Réel Embarqués Critiques 
\guillemotright\footnote{\url{https://strec.wp.imt.fr/}}. Cette UE se déroulait à TELECOM ParisTech mais 
était destinée à des élèves de Master 2 d'autres universités comme Paris 6 (Master SAR) et Paris 11 (Master 
SETI/COMASIC). Ces TPs se déroulaient pendant le premier semestre de l'année universitaire pour des 
groupes d'entre 10 et 20 personnes. Ces TPs abordaient des notions de systèmes temps-réel avec de la 
programmation RT-POSIX, ainsi que la manipulation d'outils de modélisation et génération de code.

%----------------------------------------------------------------------------------------
\subsection{Recherche}
%----------------------------------------------------------------------------------------

Mes projets de recherche ont été centrées sur des méthodes d'ordonnancement pour les systèmes 
embarqués temps-réels modernes. Les systèmes temps-réels sont souvent représentés par des tâches 
périodiques qui doivent être exécutés avant une échéance figée dans le temps. Une 
tâche temps-réel peut correspondre à un bout de code, une routine, un programme, entre autres.
\vspace{.5cm}

\textbf{I) Optimisation de la consommation énergétique pour des systèmes embarqués temps-réel}
\vspace{.5cm}

Dans le cadre de mon post-doctorat à l'INRIA de Paris, au sein de l'équipe Kopernic, je me suis intéressé à 
des problèmes de consommation énergétique pour des systèmes temps-réel. Ce problème 
est particulièrement intéressant puisque la plupart des systèmes temps-réels sont utilisés dans sur des 
architectures embarquées alimentées par des batteries et souvent déployés dans des environnements 
dangereux. Limiter la consommation énergétique tout en respectant les contraintes temporelles de ses 
systèmes représente une vrai nécessité dans le monde 
académique et industriel.

Les méthodes actuelles pour limiter la consommation énergétique sur les systèmes temps-réels sont 
souvent basées sur le changement de fréquence du processeur (en anglais Dynamic Voltage and Frequency 
Scaling : DVFS). Les processeurs modernes offrent la 
possibilité de réduire la tension fournie aux c\oe{}urs d'exécution, ce qui réduit leur fréquence d'exécution et 
diminue la consommation énergétique. Cependant, avec une fréquence moins élevée, le temps d'exécution  
des programmes augmente. Le problème consiste alors à trouver des fréquences adaptées qui réduisent le 
plus possible  la consommation en énergie tout en respectant les contraintes temporelles des programmes.


Le changement de fréquence du processur devient possible dû au fait que les systèmes temps-réels sont 
dimensionnés en prenant en compte le pire cas d'exécution pour les programmes. Néanmoins, le pire cas 
arrive dans des situation extrêmement rares et sans mécanisme de réduction énergétique, les ressources 
allouées sont gaspillées.
\vspace{.5cm}


%(Résultats obtenus)
Le projet de recherche du post-doctorat s'est organisé en deux parties. Dans un premier temps on a 
démontré, grâce des expérimentations, que l'hypothèse par rapport aux temps d'exécution des programmes 
et les fréquences du processeur n'est pas toujours correcte. En effet, les méthodes DVFS les plus répandues 
et utilisées sur des vrais systèmes d'exploitation, supposent que le temps d'exécution est complètement 
proportionnel à la fréquence du processeur. Néanmoins, ce modèle ne considère pas le fait que les 
programmes peuvent nécessiter des ressources externes au processeur, comme la mémoire ou d'autres 
périphériques.

L'objectif principal de ce projet de recherche consiste à proposer une méthode d'ordonnancement qui prend 
en compte un modèle probabiliste sur les temps d'exécution des programmes pour minimiser la 
consommation énergétique des systèmes temps-réels. En ayant connaissance sur la distribution des temps 
d'exécution des programmes grâce au modèle probabiliste, on est capable d'améliorer 


%(Diffusion résultats)

Les premiers résultats sur la variabilité du temps d'exécution  des programmes par rapport à la fréquence 
des processeurs et la mémoire ont donné lieu à un article court présenté à Real-Time Systems Symposium

(Encadrement)
\vspace{.5cm}

\textbf{II) Déploiement de systèmes à flots de données pour architectures multi-c\oe{}urs en criticité-mixte}
\vspace{.5cm}

Le projet de recherche de mon doctorat était centré sur le modèle d'exécution à criticité-mixte. Le modèle à 
criticité-mixte est un modèle temps-réel où des tâches de différentes \textit{criticitées} partagent des 
ressources matérielles. La criticité d'une tâche est définie par son importance suite à une défaillance : une 
tâche à haute criticité peut avoir des conséquences économiques ou humaines très importantes si elle est 
défaillante.

Plusieurs contributions sur l'ordonnancement pour ce type de systèmes en architectures  multi-c\oe{}urs 
existent dans la littérature, néanmoins très peu de travaux ont considéré des tâches avec des contraintes de 
précédence et ont tendance à surdimensionner les systèmes . Deux axes principaux ont été traités pendant 
mon doctorat : (i) le développement de nouvelles méthodes efficaces d'ordonnancement en criticité-mixte 
avec flots de données; et (ii) l'analyse de la disponibilité des composants les moins critiques pour ces 
systèmes. Le problème d'ordonnancement des tâches en criticité-mixte avec contraintes de précédence est 
très difficile (NP-difficile). La difficulté vient du fait que les systèmes à criticité-mixte ont besoin d'être 
ordonnancés sous différents modes d'exécution, c.à.d. que les échéances, périodes et contraintes de 
précédence doivent être respectées dans tous ces modes. À ce point, ce rajoute le fait que ces contraintes 
doivent être respectées pendant la transition des modes d'exécution.

% Résultats
\vspace{.5cm}
On a défini un modèle d'exécution appelé Mixed-Criticality Directed Acyclic Graph (MC-DAG), qui prend en 
compte notre contexte de recherche. Des approches existantes sont capables d'ordonnancer des MC-DAG 
sur des architectures multi-c\oe{}urs  mais elles ont des limitation importantes concernant leur taux 
d'acceptation (c.à.d. le nombre de lots de tâches qui peuvent être ordonnancées) et leur utilisation des 
ressources. Pour résoudre ces limitations nous avons caractérisé une condition suffisante pour avoir une 
\textit{transition de modes sûre}. Quand cette condition est respectée, le système va systématiquement 
respecter les échéances, même pendant une transition de modes. En nous appuyant sur ce résultat, 
nous avons défini une méta-heuristique pour ordonnancer des MC-DAGs sur des architectures 
multi-c\oe{}urs. Une première amélioration considérée par rapport à l'état de l'art, a été la relaxation de 
l'exécution des tâches les plus critiques. En appliquant ce principe nous avons réussi à réduire le temps de 
réponse des tâches les moins critiques, ce qui améliore le taux d'acceptation. Dans une deuxième partie, on a 
généralisé la méta-heuristique pour ordonnancer des MC-DAGs avec différentes périodes. Contrairement 
aux travaux existant, notre approche utilise moins de ressources matérielles puisque des MC-DAGs avec des 
périodes différentes peuvent partager un même c\oe{}ur d'exécution (ce qui n'était pas le cas avant). Nos 
résultats expérimentaux montrent que notre approche est beaucoup plus performante en terme de taux 
d'acceptation.

Le deuxième axe de recherche de ce projet s'intéressait à la disponible des tâches les moins critiques. Les 
approches les plus répandues dans la littérature des systèmes à criticité-mixte proposent d'arrêter 
l'exécution des tâches les moins critiques pour permettre le changement de mode de ces systèmes. Cette 
approche permet d'avoir des bons taux d'acceptation puisque moins de tâches doivent être ordonnancées 
dans les modes les plus critiques. Néanmoins, dans des systèmes embarqués temps-réels, même les tâches 
les 
moins critiques sont très 
importantes et doivent être exécutées le plus souvent possible. Nous nous sommes donc intéressés au calcul 
de la disponibilité de ces tâches moins critiques tout en conservant cette hypothèse sur le changement de 
mode et les tâches les moins critiques. On a voulu rester compatibles avec nos contributions sur 
l'ordonnancement qui visaient à améliorer les taux d'acceptation. Pour effectuer cette estimation sur la 
disponibilité, on a défini un modèle de fautes et un mécanisme de recouvrement qui permet de reprendre le 
mode d'exécution le moins critique et réincorporer les tâches les moins critiques. Avec ces informations on a 
pu voir qu'effectivement les taux de disponibilité étaient très bas ce qui pose problème dans le domaine 
industriel des systèmes temps-réel. Pour résoudre ce problème on a étendu le modèle d'exécution en 
rajoutant deux améliorations : (i) un modèle de propagation de fautes plus détaillé et (ii) l'inclusion de 
mécanismes de tolérance aux fautes.


%(Diffusion résultats)
\vspace{.5cm}
Ces travaux de recherches ont donné lieu à des articles publiés dans des conférences internationales [1-4]. 
Un papier de journal est actuellement sous revue [6]. Un framework 
open-source\footnote{\url{https://github.com/robertoxmed/MC-DAG}} a été développé pour Des 
communications internes ont été données aussi au 
sein de TELECOM ParisTech et dans des séminaires organisés par le CEA List.



\end{document}